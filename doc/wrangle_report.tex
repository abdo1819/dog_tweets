\documentclass{article}
\usepackage[utf8]{inputenc}
\usepackage{hyperref}
\usepackage{listings}
\usepackage{xcolor}



\title{wrangle report}
\author{abdo hashem}
\date{August 2020}

\begin{document}

\maketitle

\begin{abstract}
	this document is brief about main points in wrangling the data required for 
	dogs rating project, information doesn't include code but has an overview about the problems and possible solution for and related functions for detailed code refer to notebook for code implementation 
\end{abstract}

\section{gathering the data}
during data gathering there was three sources of data the goal here is to convert those data into a pandas data\_frame and save them as CSV file that can be loaded easily
\subsection{the data sources }
\begin{description}
	\item[local file] : here i used the pandas \href{https://pandas.pydata.org/pandas-docs/stable/reference/api/pandas.read_csv.html}{pandas.read\_csv} function
	\item[web files] this required the `requests` library which has functionality to download the file then used the pandas read\_csv to get the data frame taking 
	into account that we have a tsv with sepration `sep='\t'`
	\item[twiter api] here i used the `tweepy` library the loading steps are the following
	\begin{enumerate}
		\item use key from tweeter api to get authonication
		\item iterate over the tweet\_id 
		\item use tweet\_id  to get the json and save it in one line
	\end{enumerate}
	finishing the previous steps we now have tweet.txt we use this to constract a
	dataframe by making dictionary of 
	\begin{itemize}
		\item tweet\_id 
		\item number retweets
		\item number of likes
	\end{itemize}
\end{description}


\section{assessing/cleangin the twiter archive}
in assessing the data i tried to check from most to least important problems
\subsection{the rating}
\subsubsection{problems}
the dogs rating hade sevral issues caused by poor extraction and the nature of the tweet text 
those problem can be summarized as following
\begin{description}
	\item[float ratings ] some rates has decimal point for example 
	\lstinline{13.5/10} which will be reported as \lstinline{5/10}
	\item[rating like text] here some text has same form as rating for example \lstinline{24/7} or date like  \lstinline{4/20}
	\item[two ratings] this is most confusing as tweet photo has two dogs and the tweet give a rating for every one of them
	\item[joking] another tricky rating happens when giving strangely very high or low rating 
	then giving the actual rate examples at tweet\_id 835246439529840640
	\begin{quote}
		ok jomny I know you're excited 
		but 960/00 isn't a valid rating, 13/10 is tho	
	\end{quote}
\end{description}
\subsubsection{solutions}
solving those problems drives some wild choices for example the two rating case which  happened 
many times and required looking at the photos but others where much easier and had lucky general way so the cleaning went as following
\begin{description}
	
	\item[float ratings ] use regex that takes into account the decimal point \newline
	({\color{red} \lstinline{\d+\.?\d*}}
	\textcolor{green}{\lstinline{\/}}
	{\color{red} \lstinline{\d+\.?\d*}})
	\item[rating like text] this can be handled by removing those spotted rating like text.
	but luckily in all cases the real rating was written at the end so i had only to take the last 
	rate at the tweet 
	\item[two ratings] here i only took the first rating hoping it represents the recognized dog \ref{take_first}
	\item[joking] had the same luck as\emph{ rating like text} and taking the last rate was enough 
\end{description}

\subsection{dog stages}
\subsubsection{problem}
\begin{itemize}
	\item dog stages had critical tidiness issue as the column name were used to indicate the stage by giving a dog the stage name or the None if it isn't in this stage
	\item another issue appears after tidying the data that some dogs had two stages
	      or a photo has two dogs 
\end{itemize}

\subsubsection{solution}
\begin{description}
	\item[tidiness problem] using pandas \lstinline{melt} function we can get those stages into one column we can then remove those with None value 
	\item[duplication problem ] following same approach as in \cite{first_sol} we can take only the first stage mentioned as it is most likely to represent the first rated dog
\end{description}

\subsection{extra data}
those data extracted using Twitter API can be merged to the archive as it represents related
information to the tweet

\subsection{retweets and comments}
some tuples doesn't represent original tweets such as comments and retweets

\subsection{extra cleaning issues }
complex values in source column
 
\subsection{data types}
wrong data types in  timestamp which is date time  and dog stages as  categorical 
 

\section{image predictions}
\subsection{dog breed}
the main point of the image prediction data frame is to get the dog breed which was recorded 
as multilevel prediction about the content of image with confidence to each prediction and 
whither it is breed or not this represent two problems
\begin{itemize}
	\item a tidiness issue as dog breed had to be a one column with value representing 
	      the breed itself 
	\item dirty tweets that had no dogs at all in the three predictions 
	      
	      \subsubsection{solution}
	      
	      \begin{description}
	      	\item[tidiness problem] we can get the dog breed easily by getting the first prediction representing a breed as we have only three prediction
	      	\begin{lstlisting}
    1. change false breeds in `p1,p2,p3` columns with null
    2. use `combine` to get columns with non null value between `p1,p2`
    3. repeat 2 for `last_result,p3`
    4. drop useless old columns 
	      	\end{lstlisting}
	      	
	      	\item[no breed images] those can simply be dropped
	      \end{description}
	      
\end{itemize}






\end{document}